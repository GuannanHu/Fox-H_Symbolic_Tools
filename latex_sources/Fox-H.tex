% \documentclass[preview]{standalone}
\documentclass{article}
\usepackage{amsmath, url, hyperref, minted, fontspec, listings, color, amssymb}
\setmonofont{DejaVu Sans Mono}[Scale=MatchLowercase]
\newcommand{\FoxH}[5]{H_{#2}^{#1}\left(#3\:\middle\vert\: \begin{array}{l}#4\\[0.4em] #5\end{array}\right)}
\newcommand{\FoxHext}[7]{
  \renewcommand{\arraystretch}{1.5} % Adjust the factor 1.5 as needed
  H_{#2}^{#1}\left(#3\:\middle\vert\:
  \begin{array}{c|c}
    #4 & #5 \\ \hline
    #6 & #7
  \end{array}
  \right)
}
\renewcommand{\arraystretch}{1.8}

\lstset{
    keywordstyle=\color{blue},
    commentstyle=\color{green},
    stringstyle=\color{red}
}

% -------------------------------------------------
% BibTex Setup
% -------------------------------------------------
\usepackage[strict=true,style=english]{csquotes}
\usepackage[
  backend=biber,
  style=alphabetic,
  natbib=true,
  maxbibnames=10,
  maxcitenames=3,
  abbreviate=true
  ]{biblatex}
\addbibresource{./All.bib}

\begin{document}

\paragraph{Explanation:}


The purpose of this file is to include all necessary biblatex entries  for the
project  from All.bib developed by Le Chen~\cite{chen:23:spdes-bib}. The bib
file

\begin{lstlisting}[language=bash]
   Fox-H_biber.bib
\end{lstlisting}

is generated by running the following command:

\begin{lstlisting}[language=bash]
  > biber --output_format=bibtex --output_resolve Fox-H.bcf
  > biber Fox_H
\end{lstlisting}

The \textit{Fox $H$-function} plays a fundamental role in expressing the
fundamental solutions to our equations. It is a generalization of the
\textit{Meijer $G$-function} (see Chapter~16 of~\cite{olver.lozier.ea:10:nist}). 

\begin{enumerate}
  \item The ordinal paper: \cite{fox:61:g};
  \item Chapters 1 and 2 of \cite{kilbas.saigo:04:h-transforms};
  \item Section 1.12 of \cite{kilbas.srivastava.ea:06:theory};
  \item Section 8.2 of \cite{prudnikov.brychkov.ea:90:integrals};
  \item The books by Mathai and Saxea~\cite{mathai.saxena.ea:10:h-function,
    mathai.saxena:78:h-function};
  \item The book by \cite{eidelman.ivasyshen.ea:04:analytic};
  \item About this repo: \cite{chen.hu:23:some}.
\end{enumerate}

In the context of the stochastic partial differential equations (SPDEs), the Fox
$H$-function is used to express the fundamental solutions for the slow and fast
diffusion equations; see, e.g., \cite{chen.hu.ea:17:space-time},
\cite{chen.hu.ea:19:nonlinear}, \cite{chen.eisenberg:22:interpolating},
\cite{chen.hu:22:holder}, \cite{chen.guo.ea:22:moments},
\cite{mijena.nane:15:space-time}.

\printbibliography[title={References}]

\end{document}

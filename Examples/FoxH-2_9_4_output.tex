\documentclass[11pt]{article}
\usepackage{amsmath, url, hyperref, minted, fontspec}
\setmonofont{DejaVu Sans Mono}[Scale=MatchLowercase]
\newcommand{\FoxH}[5]{H_{#2}^{#1}\left(#3\:\middle\vert\: \begin{array}{l}#4\\[0.4em] #5\end{array}\right)}

\usepackage{amsmath, url, hyperref, minted, fontspec}
\setmonofont{DejaVu Sans Mono}[Scale=MatchLowercase]
\newcommand{\FoxH}[5]{H_{#2}^{#1}\left(#3\:\middle\vert\: \begin{array}{l}#4\\[0.4em] #5\end{array}\right)}
\newcommand{\FoxHext}[7]{
  \renewcommand{\arraystretch}{1.5} % Adjust the factor 1.5 as needed
  H_{#2}^{#1}\left(#3\:\middle\vert\:
  \begin{array}{c|c}
    #4 & #5 \\ \hline
    #6 & #7
  \end{array}
  \right)
}
\renewcommand{\arraystretch}{1.8}

% -------------------------------------------------
% BibTex Setup
% -------------------------------------------------
\usepackage[strict=true,style=english]{csquotes}
\setlength{\parskip}{0.5cm plus4mm minus3mm}
\usepackage[
  backend=biber,
  style=alphabetic,
  natbib=true,
  abbreviate=true
  ]{biblatex}
\addbibresource{../latex_sources/Fox-H_biber.bib}\begin{document}

\section{Example \url{FoxH-2_9_4.wls}}

\paragraph{File content}

\inputminted{text}{FoxH-2_9_4.wls}

\paragraph{Fox H-function}

\begin{align*}
  \FoxH
    {1,0}
    {0,1}
    {\cdot}
    {}
    {\left(b, \beta\right)}
\end{align*}

\begin{align*}
  \FoxHext
    {1,0}
    {0,1}
    {\cdot}
    {}
    {}
    {\left(b, \beta\right)}
    {}
\end{align*}

\paragraph{Summary}

\begin{align*}
  a^*    & = \beta \\
  \Delta & = \beta \\
  \delta & = \text{Indeterminate} \\
  \mu    & = b-\frac{1}{2} \\
  a_1^*  & = \beta \\
  a_2^*  & = 0 \\
  \xi    & = b \\
  c^*    & = \frac{1}{2} \\
\end{align*}

\paragraph{Poles}

\noindent\textbf{1. First eight poles from upper front list}

\begin{align*}
  a_{i,k} = 
  \{\}
\end{align*}
\noindent\textbf{2. First eight poles from lower front list}

\begin{align*}
  b_{j,\ell} = 
  \left(
\begin{array}{cccccccc}
 -\frac{b}{\beta } & -\frac{b+1}{\beta } & -\frac{b+2}{\beta } & -\frac{b+3}{\beta } & -\frac{b+4}{\beta } & -\frac{b+5}{\beta } & -\frac{b+6}{\beta } & -\frac{b+7}{\beta } \\
\end{array}
\right)
\end{align*}


\paragraph{Source} This example is from (2.9.4) of
~\cite{kilbas.saigo:04:h-transforms}:
\begin{align*}
  \FoxH
    {1,0}
    {0,1}
    {z}
    {}
    {\left(b, \beta\right)}
    = \frac{1}{\beta} z^{b/\beta} \exp\left(-z^{1 /\beta}\right).
\end{align*}

\printbibliography[title={References}]

\end{document}
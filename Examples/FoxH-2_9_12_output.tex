\documentclass[11pt]{article}
\usepackage{amsmath, url, hyperref, minted, fontspec}
\setmonofont{DejaVu Sans Mono}[Scale=MatchLowercase]
\newcommand{\FoxH}[5]{H_{#2}^{#1}\left(#3\:\middle\vert\: \begin{array}{l}#4\\[0.4em] #5\end{array}\right)}
\newcommand{\FoxHext}[7]{
  \renewcommand{\arraystretch}{1.5} % Adjust the factor 1.5 as needed
  H_{#2}^{#1}\left(#3\:\middle\vert\:
  \begin{array}{c|c}
    #4 & #5 \\ \hline
    #6 & #7
  \end{array}
  \right)
}
\renewcommand{\arraystretch}{1.8}

\begin{document}

\section{Example \url{FoxH-2_9_12.wls}}

\paragraph{File content}

\inputminted{text}{FoxH-2_9_12.wls}

\paragraph{Fox H-function}

\begin{align*}
  \FoxH
    {1,2}
    {2,2}
    {\cdot}
    {\left(\frac{1}{2}, 1\right), \left(\frac{1}{2}, 1\right)}
    {\left(0, 1\right), \left(-\frac{1}{2}, 1\right)}
\end{align*}

\begin{align*}
  \FoxHext
    {1,2}
    {2,2}
    {\cdot}
    {\left(\frac{1}{2}, 1\right), \left(\frac{1}{2}, 1\right)}
    {}
    {\left(0, 1\right)}
    {\left(-\frac{1}{2}, 1\right)}
\end{align*}

\paragraph{Summary}

\begin{align*}
  a^*    & = 2 \\
  \Delta & = 0 \\
  \delta & = \text{ComplexInfinity} \\
  \mu    & = -\frac{3}{2} \\
  a_1^*  & = 1 \\
  a_2^*  & = 1 \\
  \xi    & = \frac{3}{2} \\
  c^*    & = 1 \\
\end{align*}

\paragraph{Poles}

\noindent\textbf{1. First ten poles from upper front list}

\begin{align*}
  a_{i,k} = 
  \left(
\begin{array}{ccccccccccc}
 \frac{1}{2} & \frac{3}{2} & \frac{5}{2} & \frac{7}{2} & \frac{9}{2} & \frac{11}{2} & \frac{13}{2} & \frac{15}{2} & \frac{17}{2} & \frac{19}{2} & \frac{21}{2} \\
 \frac{1}{2} & \frac{3}{2} & \frac{5}{2} & \frac{7}{2} & \frac{9}{2} & \frac{11}{2} & \frac{13}{2} & \frac{15}{2} & \frac{17}{2} & \frac{19}{2} & \frac{21}{2} \\
\end{array}
\right)
\end{align*}
\noindent\textbf{2. First ten poles from lower front list}

\begin{align*}
  b_{j,\ell} = 
  \left(
\begin{array}{ccccccccccc}
 0 & -1 & -2 & -3 & -4 & -5 & -6 & -7 & -8 & -9 & -10 \\
\end{array}
\right)
\end{align*}

\end{document}
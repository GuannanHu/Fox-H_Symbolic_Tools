\documentclass{article}
\usepackage{amsmath}
\newcommand{\FoxH}[5]{H_{#2}^{#1}\left(#3\:\middle\vert\: \begin{subarray}{l}#4\\[0.4em] #5\end{subarray}\right)}
\begin{document}
\begin{align*}
\FoxH{2,3}{2,1}{\cdot}{\left(1, 1\right), \left(1, \beta\right)}{\left(\frac{d}{2}, \frac{\alpha }{2}\right), \left(1, 1\right), \left(1, \frac{\alpha }{2}\right)}
\end{align*}
\noindent\textbf{Summary}
\begin{align*}
a^* &= 2-\beta \\
\Delta &= \alpha -\beta \\
\delta &= 2^{-\alpha } \left(2^{\alpha /2} \alpha ^{\alpha /2}+\alpha ^{\alpha }\right) \beta ^{-\beta } \\
\mu &= \frac{d-1}{2} \\
a_1^* &= \frac{1}{2} (\alpha -2 \beta +2) \\
a_2^* &= 1-\frac{\alpha }{2} \\
\xi &= \frac{d}{2} \\
c^* &= \frac{1}{2} \\
\end{align*}
\noindent\textbf{Poles}\\
\noindent\textbf{1. First ten poles from upper front list}
\begin{align*}
a_{i,k} &= \left(
\begin{array}{c}
 0 \\
 1 \\
 2 \\
 3 \\
 4 \\
 5 \\
 6 \\
 7 \\
 8 \\
 9 \\
 10 \\
\end{array}
\right)
\end{align*}
\noindent\textbf{2. First ten poles from lower front list}
\begin{align*}
b_{j,\ell} &= \left(
\begin{array}{cc}
 -\frac{d}{\alpha } & -1 \\
 -\frac{d+2}{\alpha } & -2 \\
 -\frac{d+4}{\alpha } & -3 \\
 -\frac{d+6}{\alpha } & -4 \\
 -\frac{d+8}{\alpha } & -5 \\
 -\frac{d+10}{\alpha } & -6 \\
 -\frac{d+12}{\alpha } & -7 \\
 -\frac{d+14}{\alpha } & -8 \\
 -\frac{d+16}{\alpha } & -9 \\
 -\frac{d+18}{\alpha } & -10 \\
 -\frac{d+20}{\alpha } & -11 \\
\end{array}
\right)
\end{align*}
\end{document}
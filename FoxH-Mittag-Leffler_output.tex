\documentclass[preview]{standalone}
\usepackage{amsmath, url, hyperref, minted, fontspec}
\setmonofont{DejaVu Sans Mono}[Scale=MatchLowercase]
\newcommand{\FoxH}[5]{H_{#2}^{#1}\left(#3\:\middle\vert\: \begin{array}{l}#4\\[0.4em] #5\end{array}\right)}
\newcommand{\FoxHext}[7]{
  \renewcommand{\arraystretch}{1.5} % Adjust the factor 1.5 as needed
  H_{#2}^{#1}\left(#3\:\middle\vert\:
  \begin{array}{c|c}
    #4 & #5 \\ \hline
    #6 & #7
  \end{array}
  \right)
}
\renewcommand{\arraystretch}{1.8}

\begin{document}

\section{Example \url{FoxH-Mittag-Leffler.wls}}

\paragraph{File content}

\inputminted{text}{FoxH-Mittag-Leffler.wls}

\paragraph{Fox H-function}

\begin{align*}
  \FoxH
    {1,1}
    {1,2}
    {\cdot}
    {\left(0, 1\right)}
    {\left(0, 1\right), \left(1-\mu, \rho\right)}
\end{align*}

\begin{align*}
  \FoxHext
    {1,1}
    {1,2}
    {\cdot}
    {\left(0, 1\right)}
    {}
    {\left(0, 1\right)}
    {\left(1-\mu, \rho\right)}
\end{align*}

\paragraph{Summary}

\begin{align*}
  a^*    & = 2-\rho \\
  \Delta & = \rho \\
  \delta & = \text{ComplexInfinity} \\
  \mu    & = \frac{1}{2}-\mu \\
  a_1^*  & = 1 \\
  a_2^*  & = 1-\rho \\
  \xi    & = \mu -1 \\
  c^*    & = \frac{1}{2} \\
\end{align*}

\paragraph{Poles}

\noindent\textbf{1. First ten poles from upper front list}

\begin{align*}
  a_{i,k} = 
  \left(
\begin{array}{c}
 1 \\
 2 \\
 3 \\
 4 \\
 5 \\
 6 \\
 7 \\
 8 \\
 9 \\
 10 \\
 11 \\
\end{array}
\right)
\end{align*}
\noindent\textbf{2. First ten poles from lower front list}

\begin{align*}
  b_{j,\ell} = 
  \left(
\begin{array}{c}
 0 \\
 -1 \\
 -2 \\
 -3 \\
 -4 \\
 -5 \\
 -6 \\
 -7 \\
 -8 \\
 -9 \\
 -10 \\
\end{array}
\right)
\end{align*}

\end{document}
\documentclass{article}
\usepackage{amsmath}
\newcommand{\FoxH}[5]{H_{#2}^{#1}\left(#3\:\middle\vert\: \begin{subarray}{l}#4\\[0.4em] #5\end{subarray}\right)}
\begin{document}
\begin{align*}
\FoxH{1,2}{1,1}{\cdot}{\left(0, 1\right)}{\left(0, 1\right), \left(1-\mu, \rho\right)}
\end{align*}
\noindent\textbf{Summary}
\begin{align*}
a^* &= 2-\rho \\
\Delta &= \rho \\
\delta &= \text{ComplexInfinity} \\
\mu &= \frac{1}{2}-\mu \\
a_1^* &= 1 \\
a_2^* &= 1-\rho \\
\xi &= \mu -1 \\
c^* &= \frac{1}{2} \\
\end{align*}
\noindent\textbf{Poles}\\
\noindent\textbf{1. First ten poles from upper front list}
\begin{align*}
a_{i,k} &= \left(
\begin{array}{c}
 1 \\
 2 \\
 3 \\
 4 \\
 5 \\
 6 \\
 7 \\
 8 \\
 9 \\
 10 \\
 11 \\
\end{array}
\right)
\end{align*}
\noindent\textbf{2. First ten poles from lower front list}
\begin{align*}
b_{j,\ell} &= \left(
\begin{array}{c}
 0 \\
 -1 \\
 -2 \\
 -3 \\
 -4 \\
 -5 \\
 -6 \\
 -7 \\
 -8 \\
 -9 \\
 -10 \\
\end{array}
\right)
\end{align*}
\end{document}